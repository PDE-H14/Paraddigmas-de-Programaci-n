\section{Configurar Git (Terminal)}
\begin{itemize}
	\item[\textbf{\texttt{1.-}}] Abrir una terminal.
	\item[\textbf{\texttt{2.-}}] Configurar el nombre de usuario como en el ejemplo.
	\begin{lstlisting}[language=bash, caption= Ejemplo. Cambiar de dirección]
$ git config --global user.name "nombre de usuario"\end{lstlisting}
	\item[\textbf{\texttt{3.-}}] Si se creo sin archivo ``README.txt'' copiar la url (seleccionando la opción ``HTTPS'') que aparece al seleccionar el repositorio, bajo la leyenda ``Quick setup — if you’ve done this kind of thing before''. Si el repositorio se creo con un archivo ``README.txt'', seleccionar el botón verde ``Code'', y copiar la url del apartado ``HTTPS''.
	\item[\textbf{\texttt{4.-}}] Abrir una terminal.
	\item[\textbf{\texttt{5.-}}] Dirigirse a la carpeta donde se clonará el repositorio.
	\item[\textbf{\texttt{6.-}}] Ejecutar él comando ``git clone'' como se muestra abajo.
	\begin{lstlisting}[language=bash, caption= Ejemplo. Clonar repositorio]
$ git config --global user.email usuario@example.com\end{lstlisting}
\end{itemize}